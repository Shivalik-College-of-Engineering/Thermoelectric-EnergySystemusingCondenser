\documentclass[12pt]{beamer}

\begin{document}
\begin{frame}{Thermoelectric-Energy System using Condenser} 
\textbf{Introduction:         }

In today's world of industrialization global warming is also increasing side by side and hence the demand of air conditioner is increasing. In operating air conditioners electricity demand is high and due to this increase in supplies our environment is suffering more. 

\textbf{Market Research/Literature Survey:         }

The air heating of urban areas has serious consequences that can affect both human health or environment. Research on heat effects shows that electricity demand for cooling increases 1.5-2\% for every unit Fahrenheit. Urban heat islands increases overall electricity demand, as well as peak demand, when the offices and homes are running cooling systems, light and appliances.
The level of SO2 increases during winters in Trabzon, from the case study of Trabzon. There is dense air pollution in the residential area in the west of the Trabzon city. 

\end{frame}

\section{section 2}
\begin{frame}
\textbf{Hardware Requirement:         }

•	Condenser  

•	Scroll Compressor 

•	Heat Sink  

•	Peltier 

•	Ceramic Plate  

•	Power Transistor 

•	Center tap Transformer  

•	Arduino Mega  

•	LCD  (128*64) 

•	Temperature Sensor  

•	Relay 


\textbf{Software Requirement:         }

Arduino Software would be used for control the system and generate of statistical data on the screen . 


\end{frame}


\section{section 3}
\begin{frame}
\textbf{Implementation:         }

         The heat air that generates from the condenser reaches 
Thermoelectric Device through a pipe. The first end of the pipe is attached with a scroll compressor. Scroll compressor converts the low pressure air into high pressure air. The pipe is made up of copper and thermally coated from inside and this layer prevents the contact of heat air to the environment. 


[   Thermoelectric device is an electronic device that converts heat energy into electrical energy. This device are made of Aluminum heat sink, Two ceramic plates connected parallerly with a P-type or N-type semiconductor in between them. The first ceramic plate is hot side and connected to the heat sink and a copper plate. The second plate is connected with cold side. Wires are also connected with both sides , positive wire with hot side and negative wire with cold side. 
Power transistor (13009), CTC-1351 , Resistor , Center tap transformer , 
Switching circuit , Oscillation circuit and mini voltage stabilizer are all installed under the device . ]
\end{frame}


\section{section 4}
\begin{frame}

Thermoelectric device collects heat energy from the condenser and applies it on the heat sink i.e., connected with the hot side. There is a N-type or P-type region between these two plates 

•	N type contains electrons as major charge carriers 

•	P type contains holes as major charge carriers 

When heat air is applied on the sink its temperature increases and due to this the N- type or P-type region also gets heated (as the region is attached with the hot side).This sudden increase in temperature excites the electrons and holes which tends them to move towards the cold side and due to this movement a current will generate. The cold side is connected to a bulb with a wire and when the current passes through it the bulb starts glowing . 
The current produced is of very less amount (3-5v) which is not desirable that’s why a mini voltage stabilizer is assembled to increase the current output. The electricity produced by the thermoelectric device is used to charge the battery.
\end{frame}

\section{section 5}
\begin{frame}


A display screen is assembled on the top of the thermoelectric device. This display screen is used to tells us about : 

•	Temperature of heat air produced by condenser. 

•	Electricity produced by thermoelectric device. 

•	Charging status of battery in \%
 
There is an automatic relay switch system which cuts off the current supply when the battery is charged or if there is a problem in the system and after that the walls around the condenser opens to pass the heat air. 

\end{frame}

\section{section 6}
\begin{frame}

\textbf{Flowchart:         }


\begin{figure} 
\centering
\includegraphics[scale=0.6]{hello.png} 
\end{figure}

\end{frame}

\section{section 7}
\begin{frame}
\textbf{Feasibility:         }

Comfort application aim to provide a building indoor environment that remains relatively constant despite changes in external weather conditions or in internal heat loads. Air  conditioners makes deep plan buildings feasible, for otherwise they would have to be built narrower or with light wells so that inner spaces received sufficient outdoor air via natural ventilation. Air conditioners also allow Building to be taller, since wind speed increases significantly with altitude making natural ventilation for very tall buildings. Applications aim to provide a suitable environment for a process being carried out. In hot weather,  air conditioners can prevent heat smoke, dehydration for excessive sweating and other problems. Heat waves are the most lethal type of weather phenomenon in developed countries. 


\end{frame}


\section{section 7}
\begin{frame}
\textbf{Reference:         }


[1]	T. Hamad, A. Abdulhakim, Y. Hamad, J. Sheffield, Solid waste as renewable source of energy:  Case Studies in Thermal Engineering 4, p. 144. 
 
 
[2]	M. Ramadan, M. Khaled, H. El Hage, Using speed bump for power generation– experimental study, Energy Procedia 75 (2015) 867. 
 
 
[3]	M. Khaled, M. Ramadan, Heating fresh air by hot exhaust air of HVAC systems, Case Stud. Therm.  Eng. 8 (2016) 398. 
 
 
[4]	http://www.alphabetenergy.com/how-thermoelectrics-work 
 
 
[5]	https://blogs.scientificamerican.com/plugged-in/the-air-conditioner-that-makes electricity 



\end{frame}




\end{document}


